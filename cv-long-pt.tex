 \documentclass[a4paper,portuguese]{article}

\usepackage[utf8]{inputenc}
\usepackage{babel}
\usepackage[T1]{fontenc}
\pagestyle{empty}
\usepackage{graphicx}

\usepackage[pdftex,
            colorlinks=true,
            pdfstartview=FitV,
            linkcolor=blue,
            citecolor=blue,
            urlcolor=blue,]{hyperref}
\hypersetup{pdfauthor = {Hugo Peres Castilho},
            pdftitle = {Curriculum Vitae},
            pdfsubject = {Hugo Peres Castilho},
            pdfkeywords = {CV, Curriculum Vitae, Hugo Castilho, Hugo P.
Castilho, Hugo Castilho}}

\newcommand{\nome}[1]{
\hspace{-7mm}{\huge \textbf{ #1}} \vspace{-1.6mm}\\
\rule{\columnwidth}{.5mm} \vspace{-7.7mm}\\
\rule{\columnwidth}{.25mm}
}

\newcommand{\topic}[1]{
\section*{#1} \vspace{-6mm}
\rule{\columnwidth}{.25mm}
}

\begin{document}

\nome{Hugo Peres Castilho}
\begin{flushright}
    \begin{tabular}{lcl}
        \input{contact_pt.tex}
    \end{tabular}
\end{flushright}

\topic{Estudos}
\begin{itemize}
\setlength{\itemsep}{-1mm}
    \item[] Eng. Física Tecnológica
    \item[] Instituto Superior Técnico, Lisboa -- Portugal
    \item[] 1999--2001
\end{itemize}
\begin{itemize}
\setlength{\itemsep}{-1mm}
    \item[] Eng. Mecânica
    \item[] Ramo de Automação e Robótica
    \item[] Instituto Superior Técnico, Lisboa -- Portugal
    \item[] Terminado a 11/2006
\end{itemize}

\topic{Histórico Profissional}
\section*{INETI}
    \begin{tabular}{l}
        Instituto Nacional de Engenharia, Tecnologia e Inovação \\
        09/2005--05/2007 \\
        www.ineti.pt \\
        \\
        Investigação e desenvolvimento para indústria
    \end{tabular}
    \subsubsection*{CLIMA}
        Detecção de defeitos em tecidos em tempo real
        \begin{itemize}
        \setlength{\itemsep}{-1mm}
            \item[] Captura de imagens com back light e câmara linear
            \item[] Aquisição de imagem e processamento desenvolvido em C++
            \item[] Treino de algoritmos em MATLAB
            \item[] Sistema de automação Omron
            \item[] Extracção simples de features para real-time
            \item[] Redes neuronais, FFT, sistemas fuzzy, clustering
        \end{itemize}
    \subsubsection*{Controlo de Tolerância de Peças}
        Inspecção laser de segmentos de pistão
        \begin{itemize}
        \setlength{\itemsep}{-1mm}
            \item[] Profiling de peças por feixe laser
            \item[] Aquisição de imagem e processamento em C++
            \item[] Sistema de automação Omron
        \end{itemize}
    \vspace{0.5cm}
\section*{Robosavvy}
    \begin{tabular}{l}
        03/2007--12/2007 \\
        www.robosavvy.com \\
        www.ezphysics.org \\
        \\
        Plataforma de simulação para robôs humanóides \\
    \end{tabular}
    \begin{itemize}
    \setlength{\itemsep}{-1mm}
        \item[] Programação em C++
        \item[] Framework de simulação física ODE para a dinâmica de corpo rígido
        \item[] Motor gráfico 3D Ogre
    \end{itemize}
    \vspace{0.5cm}
\section*{Rose Real Estate}
    \begin{tabular}{l}
        08/2007--01/2009 \\
        rrr.pt \\
        \\
        Desenvolvimento e manutenção do website \\
    \end{tabular}
    \begin{itemize}
    \setlength{\itemsep}{-1mm}
        \item[] HTML, CSS, Javascript
        \item[] Django -- Framework de desenvolvimento web em Python
        \item[] MySQL
        \item[] Apache Webserver
    \end{itemize}
    \vspace{0.5cm}
\section*{CERN}
    \begin{tabular}{l}
        European Organization for Nuclear Research \\
        12/2007--11/2008 \\
        www.cern.ch \\
        \\
        Desenvolvimento de software, investigação \\
    \end{tabular}
    \subsubsection*{FPIAA--Finding People in Atlas Areas}
    \begin{tabular}{l}
        Sistema de monitorização de pessoal em zonas críticas do detector ATLAS \\
    \end{tabular}
    \begin{itemize}
    \setlength{\itemsep}{-1mm}
        \item[] Grelha de sensores PIR (passive infrared sensor)
        \item[] Programação em PVSS (SCADA) para interface com os sensores
        \item[] Coordenação de uma pequena equipa
    \end{itemize}
    \subsubsection*{Interface de monitorização web para ATLAS DCS (Detector Control Systems)}
    \begin{tabular}{l}
        Sistema de monitorização web para o detector ATLAS
    \end{tabular}
    \begin{itemize}
    \setlength{\itemsep}{-1mm}
        \item[] Programação em PVSS (SCADA) para recolha e display de informação de componentes do ATLAS
        \item[] Scripting em Python para monitorização do uptade de informação
    \end{itemize}
    \vspace{0.5cm}
\section*{Infolau}
    \begin{tabular}{l}
        11/2008--Presente \\
        www.infolau.com \\
        \\
        Consultadoria Bancária
    \end{tabular}
    \subsubsection*{Crédito Agrícola}
    \begin{itemize}
    \setlength{\itemsep}{-1mm}
        \item[] Desenvolvimento e manutenção do back-end de cartões
        \item[] Programação em MUMPS/PROFILE
    \end{itemize}

\topic{Publicações}
\begin{itemize}
    \item[] Hugo P. Castilho, João R. C. Pinto, António L. Serafim, "NN
Automated Defect Detection Based on Optimized Thresholding", ICIAR 2006 -
Intern. Conf. on Image Analysis and Recognition 2006, Póvoa
de Varzim, Portugal; Lecture Notes in Computer Science, Springer Verlag,
LNCS 41423, Vol.2, pp. 790-801.
    \item[] Hugo P. Castilho, Paulo J. S. Gonçalves, João R. C. Pinto, António
L. Serafim, "Intelligent Real-Time Fabric Defect Detection", ICIAR
2007, Montreal, Canada; Lecture Notes in Computer Science, Springer Verlag,
LNCS 4633, pp. 1309-1319.
    \item[] Hugo P. Castilho, António L. Serafim, João R. C. Pinto, "Automated
fabric defect detection: A real-time NN approach" , Publicação Lisboa: INETI - 
Memórias, 2006/2007, 2007.
\end{itemize}

\topic{Linguagens de Programação}
\begin{itemize}
\setlength{\itemsep}{-1mm}
    \item C/C++
    \item Python
    \item MatLab
    \item C\#
    \item Java
    \item Fortran
    \item PVSS (SCADA)
    \item MUMPS/PSL PROFILE
\end{itemize}

\topic{Aplicações/Framework}
\begin{itemize}
\setlength{\itemsep}{-1mm}
    \item Windows e Linux
    \item Subversion, CVS e git
    \item Mathematica
    \item Latex
    \item Simulink
    \item Solid Works
    \item ANSYS
    \item SQL
    \item HTML/CSS
    \item Django--Web Development Framework
    \item MIL--Biblioteca de Aquisição de Imagem
    \item Pygame--Game development python framework
    \item Box2D--2D physics engine
    \item cwiid--Wiimote library
    %\item openframeworks
    %\item OpenCV--Computer vision library
\end{itemize}

\topic{Línguas}

\begin{tabular}{llll}
Língua & Compreensão & Falado & Escrito \\
\hline
Português & Avançado & Avançado & Avançado \\
Inglês & Avançado & Avançado & Avançado \\
%Francês & Médio & Básico & Muito Básico \\
%Espanhol & Médio & Básico & Nenhum \\
\end{tabular}

\topic{Projectos Académicos/Hobby}
\begin{itemize}
    \setlength{\itemsep}{-1mm}
    \item[] {\bf Detecção de defeitos em tecidos em tempo real (tese):}\\ \vspace{-7mm}
    \begin{itemize}
        \setlength{\itemsep}{-1mm}
        \item[] Redes neuronais, sistemas e inferência fuzzy, clustering
        \item[] Aquisição e processamento de imagem
        \item[] C++, Matlab, MIL
    \end{itemize}

    \item[] {\bf  Plataforma de optimização distribuída para gestão de uma supply-chain:}\\ \vspace{-7mm}
    \begin{itemize}
        \setlength{\itemsep}{-1mm}
        \item Optimização por colónia de formigas
        \item Java, Matlab, interface Java--Matlab, plataforma servidor/cliente TCP/IP
    \end{itemize}

    \item[] {\bf  Optimização de produção com tempos de setup para um shop floor organizado por células:}\\ \vspace{-7mm}
    \begin{itemize}
    \setlength{\itemsep}{-1mm}
        \item Optimização por colónia de formigas
        \item MATLAB
    \end{itemize}

    \item[] {\bf  Gestão e controlo de uma linha de produção:}\\ \vspace{-7mm}
    \begin{itemize}
    \setlength{\itemsep}{-1mm}
        \item Plataforma servidor/cliente UDP e TCP
        \item Linguagem de gestão de linha
        \item C\#
    \end{itemize}

    \item[] {\bf Tracking e visual servoing:}\\ \vspace{-7mm}
    \begin{itemize}
        \setlength{\itemsep}{-1mm}
        \item C++, MIL
    \end{itemize}

    \item[] {\bf Programação de PLCs:}\\ \vspace{-7mm}
    \begin{itemize}
        \setlength{\itemsep}{-1mm}
        \item PCD (SAIA Burgess) e Zelio Logic (Schneider Electric)
        \item Instruction language, grafcet e ladder logic
    \end{itemize}

    \item[] {\bf  Programação e simulação de um braço robótico por ACL:}\\ \vspace{-7mm}
    \begin{itemize}
    \setlength{\itemsep}{-1mm}
        \item Simulação em VRML e interpretador de  ACL
        \item Comunicação série por ACL com o braço robótico
    \end{itemize}

    \item[] {\bf Sistema de Cultura Hidropônico:}\\ \vspace{-7mm}
    \begin{itemize}
    \setlength{\itemsep}{-1mm}
        \item Construção e electrónica DIY
    \end{itemize}

    \item[] {\bf Jogo Python Wiimote:}\\ \vspace{-7mm}
    \begin{itemize}
    \setlength{\itemsep}{-1mm}
        \item Pygame, python-box2d, cwiid
    \end{itemize}

    \item[] {\bf Btrfs raid server:}\\ \vspace{-7mm}
    \begin{itemize}
    \setlength{\itemsep}{-1mm}
        \item Btrfs filesystem em raid10
    \end{itemize}
\end{itemize}

\end{document}
